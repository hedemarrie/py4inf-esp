% The contents of this file is 
% Copyright (c) 2009-  Charles R. Severance, All Righs Reserved

\chapter{?`Por qu\'e se debe aprender a programar?}

Escribir programas (o programar) es una actividad muy creativa y gratificante. Usted puede escribir programas para ganarse la vida o para resolver un problema de an\'alisis dif\'icil o, simplemente por el placer de poder ayudar a alguien a enontrar soluciones. Este libro asume que \textbf{todos} necesitan saber c\'omo programar y, que una vez que se aprende, se descubre lo que se quiere lograr con la nueva destreza adquirida.  

Estamos rodeados en nuestra vida diaria de computadoras que van desde p\'ortatiles a tel\'efonos celulares. Se puede pensar de estas computadoras como si fueran nuestros ``asistentes personales'', quienes realizan muchas de las tareas en nuestro lugar. El \textit{hardware} en las computadoras de hoy est\'an construidas esencialmente para que continuamente hagan la pregunta, 
``?`Qu\'e quieres que haga despu\'es?''.

\beforefig
\centerline{\includegraphics[height=1.00in]{figs2/pda.eps}}
\afterfig

Los programadores agreagan un sistema operativo y una serie de aplicaciones al \textit{hardware} y nosotros terminamos con un Asistente Personal Digital muy \'util para diferentes cosas.

Nuestras computadoras son r\'apidas y tienen una vasta cantidad de memoria y podr\'ian ser mucho m\'as si tan solo supi\'eramos el lenguage para hablarles y explicarles lo que quisi\'eramos que ``hagan despu\'es''. Si supi\'eramos este lenguage, podr\'iamos decirle a la computadora que haga tareas repetitivas en nuestro lugar. Interesantemente, el tipo de cosas que las computadoras hacen mejor, a menudo, son cosas que los humanos encontramos aburridas y abrumadoras.

Por ejemplo, mire los tres primeros p\'arrafos de este cap\'itulo y d\'igame cu\'al es la palabra m\'as com\'un. Mientras que usted es capaz de leer y comprender las palabras en cuesti\'on de segundos, contarlas es doloroso porque no es el tipo de problema que la mente humana ha sido dise\~nada para resolver. Para una computadora es todo lo contrario, leer y comprender el texto de un pedazo de papel es dif\'icil para una computadora, pero contar las palabras y decirle a usted cu\'antas veces la palabra m\'as com\'un fue usada es muy f\'acil para la computadora:

\beforeverb
\begin{verbatim}
python palabras.py
Enter file:palabras.txt
de 7
\end{verbatim}
\afterverb
%
Nuestro ``asistente personal de an\'alisis de informaci\'on'' nos dice r\'apidamente que la palabra ``de'' aparece 7 veces en los primeros tres p\'arrafos de este cap\'itulo.

El hecho de que las computadoras sean tan buenas en lo que los humanos no lo son es la raz\'on por la cual usted debe desarrollar la destreza de hablar el ``lenguage de computadora''. Una vez que aprenda este lenguage ver\'a c\'omo puede delegar tareas miniales a su socia (la computadora), liberando as\'i su tiempo para hacer aquellas cosas para las que ha sido creado de manera especial. Usted aporta creatividad, intuici\'on e inventividad a esta relaci\'on colaborativa.  

\section{Creatividad y motivaci\'on}

Aunque este libro no est\'a dirigido a programadores profesionales, la programaci\'on puede ser un trabajo muy gratificante tanto personal como financieramente. 
Elaborar programas \'utiles, elegantes e inteligentes para el beneficio de otros es una actividad muy creativa. Su computadora o Asistente Personal Digital (PDA por sus siglas en ingl\'es) 
generalmente contiene diferentes programas de diferentes programadores, cada uno de ellos compitiendo por su atenci\'on e inter\'es. Ellos tratan de hacer lo mejor para satisfacer sus necesidades y proveerle una excelente experiencia de usuario, en el proceso. En algunas situaciones, cuando usted escoge una aplicaci\'on (software), los programadores son directamente compensados en virtud de su elecci\'on.

Al considerar los programas como obras creativas de grupos de programadores, quiz\'a la siguiente imagen sea una versi\'on m\'as sensible de nuestro PDA:

\beforefig
\centerline{\includegraphics[height=1.00in]{figs2/pda2.eps}}
\afterfig

Por ahora, nuestra motivaci\'on primordial no es hacer dinero o complacer a los usuarios, sino hacernos m\'as productivos en el manejo de datos e informaci\'on que encontraremos en nuestra vida diaria.
Cuando usted comience, usted es tanto programador como usuario de sus propios programas. En la medida en que adquiera m\'as destreza como programador y programar se sienta como una actividad m\'as y m\'as creativa para usted, se le ocurrir\'an ideas sobre el desarrollo de programas para otros.

\section{Arquitectura del hardware de la computadora}
\index{hardware}
\index{hardware!arquitectura}

Antes de comenzar a aprender el lenguage necesario para darle instrucciones a las computadoras para desarrollar software, es necesario aprender una peque\~na cantidad sobre c\'omo se construyen las computadoras. Si usted desbaratara su computadoras o tel\'efono celular y mirara lo que hay dentro, encontar\'ia las siguientes partes:

\beforefig
\centerline{\includegraphics[height=2.50in]{figs2/arch.eps}}
\afterfig

Las definiciones funcionales de alto nivel de estas partes son las siguientes:

\begin{itemize}

\item La {\bf Unidad Central de Procesamiento} conocida por sus siglas en ingl\'es como CPU es esa parte de la computadora que ha sido construida para obsesionarse con la cuesti\'on ``?`qu\'e hay que hacer despu\'es?''. Si su computadora tiene una capacidad de 3.0 Gigahertz, eso significa que el CPU preguntar\'a ``?`qu\'e hay que hacer despu\'es?''
tres mil millones de veces por segundo. Usted va a tener que aprender c\'omo hablar r\'apido para mantener el ritmo de la CPU.

\item La {\bf Memoria Principal} se usa para almacenar la informaci\'on
que la CPU necesita de af\'an. La memoria principal es casi tan r\'apida como la CPU, pero la informaci\'on almacenada en la memoria principal desaparece cuando se apaga la computadora.

\item La {\bf Memoria Secundaria} tambi\'en se usa para almacenar informaci\'on, pero es mucho m\'as lenta que la memoria principal.
La ventaja de la memoria secundaria es que puede almacenar informaci\'on aun cuando la computadora est\'e apagada. Ejemplos de la memoria secundaria son los discos duros o memorias flash (esto se encuentran t\'ipicamente en dispositivos USB y reproductores de m\'usica portables).

\item Los {\bf Dispositivos de entrada y salida (Input y Output)} son su monitor, teclado, mouse, microf\'ono, parlantes, tablero t\'actil integrado, etc.  
Estos representan todas las formas en que interactuamos con la computadora.

\item Hoy en d\'ia la mayor\'ia de las computadoras tambi\'en tienen
{\bf Conexi\'on a la Red (Network Connection)} para recibir informaci\'on por medio de una red.
Se puede pensar de una red como un lugar muy lento para almacenar y recuperar datos que pordr\'ian no estar siempre ``actualizados''. En cierto sentido, la red es m\'as lenta y a veces una forma poco fiable de
{\bf Memoria Secundaria}.
\end{itemize}

Mientras que la mayor\'ia de los detalles sobre c\'omo trabajan estos componentes es mejor dej\'arselo a los  que construyen computadoras, conocer la terminolog\'ia ayuda para podernos referir a estos componentes cuando escribimos nuestros programas.

Como programador, su trabajo es utilizar y coordinar cada uno de estos recursos para dar soluci\'on a los problemas para los cuales usted necesita analizar los datos que necesita usar. Como programador usted va a ``hablarle'' la mayor\'ia del tiempo a la CPU para decirle ``qu\'e debe hacer despu\'es''. Algunas veces usted le va a decir a la CPU que utilice la memoria principal, la memoria secundaria, la red o los dispositivos de entrada/salida (input/output).

\beforefig
\centerline{\includegraphics[height=2.50in]{figs2/arch2.eps}}
\afterfig

Usted tiene que ser la persona que le responda a la CPU la pregunta ``?`Qu\'e debe hacer despu\'es?'', pero ser\'ia muy inc\'omodo reducirlo a usted a un tama\~no de 5mm e insertarlo en la computadora solo para poder emitir un comando tres mil millones de veces por segundo. As\'i que en vez de hacer eso, usted debe escribir sus instrucciones de antemano.
A estas instrucciones almancenadas les llamamos {\bf programa} y al acto de escribir estas instrucciones y escribirlas correctamente, {\bf programaci\'on}.

\section{Comprender el concepto de programaci\'on}

En el resto de este libro vamos a tratar de convertirlo a usted en una persona
con destreza en el arte de programaci\'on. Al final, usted ser\'a un 
{\bf programador} --- quiz\'a no un programador profesional, pero por lo menos tendr\'a las destrezas para mirar un problema de an\'alisis de informaci\'on y datos y desarrollar un programa como soluci\'on al problema.

\index{resoluci\'on de problemas}

En un sentido, usted necesita dos destrezas para ser programador:

\begin{itemize}

\item Primero, usted necesita saber el lenguage de programaci\'on (Python) -
usted necesita saber el vocabulario y la gram\'atica. Usted necesita ser capaz de deletrear las palabras en este nuevo lenguage apropiadamente y saber c\'omo construir ``oraciones'' bien formadas en este nuevo lenguage.

\item Segundo, usted necesita ``contar una historia''. Al escribir una historia,
usted combina palabras y oraciones para transmitirle al lector una idea. Componer una historia requiere de destreza y arte, y la destreza de escribir se mejora escribiendo y recibiendo retroalimentaci\'on. En programaci\'on, nuestro programa es la ``historia'' y el problema que usted trata de resolver es la ``idea''.

\end{itemize}

Una vez que usted aprende un programa como Python, encontrar\'a mucho m\'as f\'acil aprender un segundo lenguage de programaci\'on como JavaScript o C++. El nuevo lenguage de programaci\'on tiene un vocabulario y gram\'atica muy diferente, pero una vez que adquiera destreza en la resoluci\'on de problemas, las destrezas van a ser las mismas para todos los lenauages de programaci\'on.

Usted aprender\'a el ``vocabulario'' y ``oraciones'' de Python bien r\'apido.
Le tomar\'a m\'as tiempo aprender a escribir un programa coherente para resolver un problema nuevo. Ense\~namos programaci\'on como ense\~namos escritura. Empezamos leyendo y explicando programas y despu\'es escribimos programas sencillos para luego escribir programas m\'as complejos, en el transcurso del tiempo.
En alg\'un momento usted encuentra ``su inspiraci\'on'' y comienza a descubrir patrones usted mismo y puede ver de manera m\'as natural c\'omo tomar un problema y escribir un programa para darle una soluci\'on, y una vez que llegue a ese punto, la programaci\'on se convierte en un proceso agradable y creativo.  

Comenzamos con el vocabulario y la estructura de los programas de Python. Tenga paciencia cuando la sencillez de los ejemplos le recuerden cuando comenz\'o a leer por primera vez. 

\section{Palabras y oraciones}
\index{lenguage de programaci\'on}
\index{lenguage!programaci\'on}

A diferencia del lenguage humano, el vocabulario de Python es realmente peque\~no. A este ``vocabulario''se le llama ``palabras reservadas''. Estas son palabras que tienen un significado muy especial para Python. Cuando Python ve estas palabras en un programa de Python, ellas tienen un solo significado para Python. M\'as adelante, en la medida en que usted vaya escribiendo programas, usted va a crear sus propias palabras, que guardan un significado espec\'ifico para usted, llamadas {\bf variables}. Usted tendr\'a una gran latitud al escoger nombres para sus variables, pero no podr\'a usar ninguna de las palabras reservadas para Python como nombre de variables.

En un sentido, cuando entrenamos un perro, usamos palabras especiales como ``si\'entese'', ``quedese ah\'i'', y ``recoja''. Tambi\'en, cuando usted le habla a un perro y no utiliza ninguna de estas palabras reservadas, el perro lo mira con una mirada de asombro en la cara hasta que usted diga una de las palabras reservadas.  
Por ejemplo, si usted dice ``Ojal\'a que m\'as gente camine para mejorar su salud'', 
lo m\'as seguro que escucha un perro es ``bla bla bla {\bf camine} bla bla bla bla.''
Esto es debido a que ``camine'' es una palabra reservada en el lenguage de un perro. Muchos sugerir\'ian que el lenguage entre humanos y gatos no tiene palabras reservadas\footnote{\url{http://xkcd.com/231/}}.

Las palabras reservadas en el lenguage con que los humanos le hablan a 
Python incluye las siguientes:

\beforeverb
\begin{verbatim}
and       del       from      not       while    
as        elif      global    or        with     
assert    else      if        pass      yield    
break     except    import    print              
class     exec      in        raise              
continue  finally   is        return             
def       for       lambda    try
\end{verbatim}
\afterverb
%
Eso es todo, y a diferencia del perro, Python ya est\'a completamente entrenado.
Cuando usted dice ``try'', Python va a tratar cada vez que usted lo diga sin fallar.

Aprenderemos estas palabras reservadas y c\'omo se usan en el momento apropiado, pero por ahora nos enfocaremos en el equivalente al ``idioma'' de Python en t\'erminos relacionados al leguage como vimos entre la comunicaci\'on humano-perro en el ejemplo anterior. Lo bueno de pedirle a Python que hable es que podemos aun decirle qu\'e decir d\'andole un mensaje entre comillas:

\beforeverb
\begin{verbatim}
print 'Hello world!'
\end{verbatim}
\afterverb

Y ya hemos escrito nuestra primera oraci\'on sit\'acticamente correcta en Python.
Nuestra oraci\'on comienza con la palabra reservada {\bf print} seguida de una cadena \textit{string} que nosotros mismos escojamos, encerrada en comillas simples.

\section{Conversaci\'on con Python}

Ahora que ya sabemos una palabra y una oraci\'on sencilla  en Python,
necesitamos saber c\'omo se comienza una conversaci\'on con Python para probar 
nuestras destrezas de lenguage.

Antes de que pueda conversar con Python, usted primero debe instalar la aplicaci\'on de Python
(software) en su computadora y aprender c\'omo iniciar Python en ella. Esto es demasiado detalle para este cap\'itulo, as\'i que le sugiero que consulte \url{www.pythonlearn.com} donde he detallado instrucciones e im\'agenes de pantalla para mostrar c\'omo se instala Python 
en los sistemas Macintosh y Windows. En alg\'un momento, usted va a estar en 
una terminal o ventana de comando y va a escribir {\bf python} y el int\'erprete 
de Python va a comenzar a ejecutar los comandos en modo interactivo 
y le va a aparecer algo as\'i:
\index{interactive mode}

\beforeverb
\begin{verbatim}
Python 2.6.1 (r261:67515, Jun 24 2010, 21:47:49) 
[GCC 4.2.1 (Apple Inc. build 5646)] on darwin
Type "help", "copyright", "credits" or "license" for more information.
>>> 
\end{verbatim}
\afterverb
%
El indicador {\tt >>>} es la manera en que el int\'erprete de Python le pregunta a usted ``?`Qu\'e quiere que haga despu\'es?''. Python est\'a listo para tener una conversaci\'on con usted. Lo que tiene que saber es c\'omo hablar en el lenguage de Python y usted podr\'a sostener una conversaci\'on.

Vamos a decir por ejemplo que usted no sab\'ia ni la m\'as m\'inima palabra u oraci\'on. Es posible que quisiera utilizar la frase est\'andar que usan los astronautas cuando aterrizan en un planeta lejano y tratan de hablar con los habitantes de ese planeta:

\beforeverb
\begin{verbatim}
>>> Yo vengo en plan de paz, favor llevarme a su autoridad 
  File "<stdin>", line 1
    Yo vengo en plan de paz, favor llevarme a su autoridad
         ^
SyntaxError: invalid syntax
>>> 
\end{verbatim}
\afterverb
%
Esto no va bien. A menos que se le ocurra algo r\'apido, los habitantes de ese planeta lo pueden atacar con sus espadas, ensartarlo en un palo, cocinarlo en fuego, y com\'erselo a la cena.

Por fortuna, usted llevaba una copia de este libro en su viaje y buscando lleg\'o exactamente a esta p\'agina e intent\'o nuevamente:

\beforeverb
\begin{verbatim}
>>> print 'Hello world!'
Hello world!
\end{verbatim}
\afterverb
%
Esto se ve mucho mejor, as\'i que trata de comunicarse un poco m\'as:

\beforeverb
\begin{verbatim}
>>> print 'Usted debe ser el legendario dios que viene del cielo'
Usted debe ser el legendario dios que viene del cielo
>>> print 'Los hemos estado esperando por mucho tiempo'
Los hemos estado esperando por mucho tiempo
>>> print 'Nuestra leyenda dice que ustedes saben muy rico con mostaza'
Nuestra leyenda dice que ustedes saben muy rico con mostaza
>>> print 'Tendremos un banquete esta noche a menos que ustedes digan
  File "<stdin>", line 1
    print 'Tendremos un banquete esta noche a menos que ustedes digan
                                                     ^
SyntaxError: EOL while scanning string literal
>>> 
\end{verbatim}
\afterverb
%
La conversaci\'on iba bien, pero cometi\'o un errorcito al usar el lenguage de Python y Python 
le sac\'o las u\~nas.

A este punto, usted debe haber ca\'ido en cuenta que aunque Python 
es inmensamente complejo y poderoso, y muy quisquilloso en relaci\'on a la sintaxis con la que se debe comunicar con \'el, Python {\em 
no} es inteligente. Usted est\'a sosteniendo una conversaci\'on con usted mismo, pero utilizando una sintaxis apropiada.

En este sentido, cuando usted usa un programa escrito por otra persona, la conversaci\'on se da entre usted y esos otros programadores con Python actuando como intermediario.  Python
es una manera en que los creadores de programas expresan c\'omo se supone que la conversaci\'on debe proceder. Y en tan s\'olo unos cuantos cap\'itulos m\'as usted ser\'a uno de esos programadores que usan Python para hablar con los usuarios de su programa.

Antes de cerrar nuestra primera conversaci\'on con el int\'erprete de Python, usted probablemente deba aprender cu\'al es la manera apropiada de decir ``good-bye'' cuando interacciona con los habitantes del Planeta Python:

\beforeverb
\begin{verbatim}
>>> good-bye
Traceback (most recent call last):
  File "<stdin>", line 1, in <module>
NameError: name 'good' is not defined

>>> if you don't mind, I need to leave
  File "<stdin>", line 1
    if you don't mind, I need to leave
             ^
SyntaxError: invalid syntax

>>> quit()
\end{verbatim}
\afterverb
%
Usted notar\'a que el error es diferente en los dos primeros intentos. El segundo error es diferente porque 
{\bf if} es una palabra reservada y Python vi\'o la palabra reservada y pens\'o que usted estaba tratando de decirle algo, pero la sintaxis de la oraci\'on no era correcta.

La manera apropiada de decirle ``good-bye'' a Python es escribiendo  
{\bf quit()} en los s\'imbolos de la terminal que aparecen as\'i {\tt >>>}.
Le hubiera tomado bastante adivinar eso, as\'i que tener el libro a la mano probablemente le resultar\'a \'util.

\section{Terminolog\'ia: int\'erprete y compilador}

Python es un lenguage de {\bf alto nivel} cuya intenci\'on es hacerlo relativamente sencillo para ser le\'ido y escrito por humanos y para que las computadoras 
lo lean y lo procesen. Otros lenguages de alto nivel incluyen Java, C++,
PHP, Ruby, Basic, Perl, JavaScript, y muchos otros. El hardware dentro de la Unidad
Central de Procesamiento (CPU por sus siglas en ingl\'es) no entiende ninguno de estos lenguages de alto nivel.

La CPU comprende un lenguage que se conoce como {\bf machine-language} o lenguage de m\'aquina. El Lenguage de M\'aquina es muy simple y francamente muy agotador para escribir porque se representa por ceros y unos:

\beforeverb
\begin{verbatim}
01010001110100100101010000001111
11100110000011101010010101101101
...
\end{verbatim}
\afterverb
%
El Lenguage de M\'aquina se ve muy sencillo por encima, dado que consiste solo de ceros y unos, pero su sintaxis es aun muy compleja y mucho m\'as elaborada que Python. As\'i que muy pocos programadores escriben en lenguage de m\'aquina. En lugar de eso, lo que hacemos es construir varios traductores que le permitan a los programadores escribir en lenguages de alto nivel como Python o JavaScript
y estos traductores convierten los programas a lenguage de m\'aquina para la ejecuci\'on de los programas en la CPU.

Siendo que el lenguage de m\'aquina est\'a asociado al hardware de la computadora, el lenguage de m\'aquina no es {\bf portable} a trav\'es de los diferentes tipos de hardware. Los Programas escritos en lenguages de 
alto nivel se pueden mover entre las diferentes computadoras usando un int\'erprete diferente en la nueva computadora o al recopilar el c\'odigo para crear una versi\'on de lenguage de m\'aquina del programa que se pretende usar en otra computadora diferente.

Estos int\'erpretes de lenguages de programaci\'on se ubican en dos categorias generales:
(1) int\'erpretes y (2) compiladores.

Un {\bf int\'erprete} lee el c\'odigo fuente del programa como ha sido escrito por el
programador, analiza el c\'odigo e interpreta las instrucciones al instante.
Python es un int\'erprete y cuando estamos ejecutando la aplicaci\'on Python, interactivamente, 
podemos escribir una l\'inea de Python (una oraci\'on) y Python la procesa de inmediato,
y queda listo para que nosotros escribamos otra l\'inea en Python.   

Algunas de las l\'ineas en Python le dicen a Python que usted quiere que recuerde algunos valores para m\'as tarde. Necesitamos escoger un nombre para almacenar ese valor en memoria y podemos usar ese nombre simb\'olico para recuperar el valor m\'as tarde. Usamos el t\'ermino {\bf variable} para referirnos al nombre con el cual se almacenan esos datos.

\beforeverb
\begin{verbatim}
>>> x = 6
>>> print x
6
>>> y = x * 7
>>> print y
42
>>> 
\end{verbatim}
\afterverb
%
En este ejemplo, le pedimos a Python que recuerde el valor seis y que se lo asigne a {\bf x}
de modo que podamos recuperar el valor m\'as tarde. Verificamos que en efecto Python ha recordado el valor usando {\bf print}. Luego le pedimos a Python que recupere {\bf x} y lo multiplique por siete y que ponga el nuevo valor asign\'andoselo a {\bf y}. Despu\'es le pedimos a Python que imprima el valor actual en {\bf y}.

Aunque estamos escribiendo estos comandos en Python una l\'inea a la vez, Python
los trata como una secuencia ordenada de oraciones y es capaz de recuperar datos creados en frases escritas desde el principio. Estamos escribiendo nuestro primer p\'arrafo sencillo con cuatro oraciones en un orden l\'ogico y significativo.

La naturaleza de un {\bf int\'erpreter} es ser capaz de mantener una conversaci\'on interactiva como se muestra arriba. Un {\bf compilador} necesita que se le entregue el programa entero en un archivo y luego lo procesa para traducirlo al c\'odigo de fuente de alto nivel en lenguage de m\'aquina, despu\'es, el compilador pone el lenguage de m\'aquina resultante en un archivo para ser ejecutado m\'as tarde.

Si usted tiene un sistema Windows, a menudo estos programas ejecutables de lenguage de m\'aquina tienen un sufijo ``.exe'' o ``.dll'' que significa ``ejecutable'' y ``biblioteca dinamicamente cargable'', respectivamente. En Linux y Macintosh no hay sufijo que indentifique un archivo como ejecutable.

Si usted abriera un archivo ejecutable en un editor de texto, ver\'ia algo completamente ilegible:

\beforeverb
\begin{verbatim}
^?ELF^A^A^A^@^@^@^@^@^@^@^@^@^B^@^C^@^A^@^@^@\xa0\x82
^D^H4^@^@^@\x90^]^@^@^@^@^@^@4^@ ^@^G^@(^@$^@!^@^F^@
^@^@4^@^@^@4\x80^D^H4\x80^D^H\xe0^@^@^@\xe0^@^@^@^E
^@^@^@^D^@^@^@^C^@^@^@^T^A^@^@^T\x81^D^H^T\x81^D^H^S
^@^@^@^S^@^@^@^D^@^@^@^A^@^@^@^A\^D^HQVhT\x83^D^H\xe8
....
\end{verbatim}
\afterverb
%
It is not easy to read or write machine language so it is nice that we have
{\bf interpreters} and {\bf compilers} that allow us to write in a high-level
language like Python or C.

Now at this point in our discussion of compilers and interpreters, you should 
be wondering a bit about the Python interpreter itself.  What language is 
it written in?  Is it written in a compiled language?  When we type
``python'', what exactly is happening?

The Python interpreter is written in a high level language called ``C''.  
You can look at the actual source code for the Python interpreter by
going to \url{www.python.org} and working your way to their source code.
So Python is a program itself and it is compiled into machine code and
when you installed Python on your computer (or the vendor installed it),
you copied a machine-code copy of the translated Python program onto your
system.   In Windows the executable machine code for Python itself is likely
in a file with a name like:

\beforeverb
\begin{verbatim}
C:\Python27\python.exe
\end{verbatim}
\afterverb
%
That is more than you really need to know to be a Python programmer, but
sometimes it pays to answer those little nagging questions right at 
the beginning.

\section{Escribir un programa}

Typing commands into the Python interpreter is a great way to experiment 
with Python's features, but it is not recommended for solving more complex problems.

When we want to write a program, 
we use a text editor to write the Python instructions into a file,
which is called a {\bf script}.  By
convention, Python scripts have names that end with {\tt .py}.

\index{script}

To execute the script, you have to tell the Python interpreter 
the name of the file.  In a Unix or Windows command window, 
you would type {\tt python hello.py} as follows:

\beforeverb
\begin{verbatim}
csev$ cat hello.py
print 'Hello world!'
csev$ python hello.py
Hello world!
csev$
\end{verbatim}
\afterverb
%
The ``csev\$'' is the operating system prompt, and the ``cat hello.py'' is 
showing us that the file ``hello.py'' has a one line Python program to print
a string.

We call the Python interpreter and tell it to read its source code from
the file ``hello.py'' instead of prompting us for lines of Python code
interactively.

You will notice that there was no need to have {\bf quit()} at the end of
the Python program in the file.   When Python is reading your source code
form a file, it knows to stop when it reaches the end of the file.

\section{?`Qu\'e es un programa?}

The definition of a {\bf program} at its most basic is a sequence
of Python statements that have been crafted to do something.
Even our simple {\bf hello.py} script is a program.  It is a one-line
program and is not particularly useful, but in the strictest definition,
it is a Python program.

It might be easiest to understand what a program is by thinking about a problem 
that a program might be built to solve, and then looking at a program
that would solve that problem.

Lets say you are doing Social Computing research on Facebook posts and 
you are interested in the most frequently used word in a series of posts.
You could print out the stream of facebook posts and pore over the text
looking for the most common word, but that would take a long time and be very 
mistake prone.  You would be smart to write a Python program to handle the
task quickly and accurately so you can spend the weekend doing something 
fun.

For example look at the following text about a clown and a car.  Look at the 
text and figure out the most common word and how many times it occurs.

\beforeverb
\begin{verbatim}
the clown ran after the car and the car ran into the tent 
and the tent fell down on the clown and the car 
\end{verbatim}
\afterverb
%
Then imagine that you are doing this task looking at millions of lines of 
text.  Frankly it would be quicker for you to learn Python and write a 
Python program to count the words than it would be to manually 
scan the words.

The even better news is that I already came up with a simple program to 
find the most common word in a text file.  I wrote it,
tested it, and now I am giving it to you to use so you can save some time.

\beforeverb
\begin{verbatim}
name = raw_input('Enter file:')
handle = open(name, 'r')
text = handle.read()
words = text.split()
counts = dict()

for word in words:
   counts[word] = counts.get(word,0) + 1

bigcount = None
bigword = None
for word,count in counts.items():
    if bigcount is None or count > bigcount:
        bigword = word
        bigcount = count

print bigword, bigcount
\end{verbatim}
\afterverb
%
You don't even need to know Python to use this program.  You will need to get through 
Chapter 10 of this book to fully understand the awesome Python techniques that were
used to make the program.  You are the end user, you simply use the program and marvel
at its cleverness and how it saved you so much manual effort.
You simply type the code 
into a file called {\bf words.py} and run it or you download the source 
code from \url{http://www.pythonlearn.com/code/} and run it.

\index{program}
This is a good example of how Python and the Python language are acting as an intermediary
between you (the end-user) and me (the programmer).  Python is a way for us to exchange useful
instruction sequences (i.e. programs) in a common language that can be used by anyone who 
installs Python on their computer.  So neither of us are talking {\em to Python},
instead we are communicating with each other {\em through} Python.

\section{Los bloques de construcci\'on de un programa}

In the next few chapters, we will learn more about the vocabulary, sentence structure,
paragraph structure, and story structure of Python.  We will learn about the powerful
capabilities of Python and how to compose those capabilities together to create useful
programs.

There are some low-level conceptual patterns that we use to construct programs.  These
constructs are not just for Python programs, they are part of every programming language
from machine language up to the high-level languages.

\begin{description}

\item[input:] Get data from the ``outside world''.  This might be 
reading data from a file, or even some kind of sensor like 
a microphone or GPS.  In our initial programs, our input will come from the user
typing data on the keyboard.

\item[output:] Display the results of the program on a screen
or store them in a file or perhaps write them to a device like a
speaker to play music or speak text.

\item[sequential execution:] Perform statements one after
another in the order they are encountered in the script.

\item[conditional execution:] Check for certain conditions and
execute or skip a sequence of statements.

\item[repeated execution:] Perform some set of statements 
repeatedly, usually with
some variation.

\item[reuse:] Write a set of instructions once and give them a name
and then reuse those instructions as needed throughout your program.

\end{description}

It sounds almost too simple to be true and of course it is never
so simple.  It is like saying that walking is simply
``putting one foot in front of the other''.  The ``art'' 
of writing a program is composing and weaving these
basic elements together many times over to produce something
that is useful to its users.

The word counting program above directly uses all of 
these patterns except for one.

\section{?`Qu\'e podr\'ia salir mal?}

As we saw in our earliest conversations with Python, we must
communicate very precisely when we write Python code.  The smallest
deviation or mistake will cause Python to give up looking at your
program.

Beginning programmers often take the fact that Python leaves no
room for errors as evidence that Python is mean, hateful and cruel.
While Python seems to like everyone else, Python knows them 
personally and holds a grudge against them.  Because of this grudge,
Python takes our perfectly written programs and rejects them as 
``unfit'' just to torment us.

\beforeverb
\begin{verbatim}
>>> primt 'Hello world!'
  File "<stdin>", line 1
    primt 'Hello world!'
                       ^
SyntaxError: invalid syntax
>>> primt 'Hello world'
  File "<stdin>", line 1
    primt 'Hello world'
                      ^
SyntaxError: invalid syntax
>>> I hate you Python!
  File "<stdin>", line 1
    I hate you Python!
         ^
SyntaxError: invalid syntax
>>> if you come out of there, I would teach you a lesson
  File "<stdin>", line 1
    if you come out of there, I would teach you a lesson
              ^
SyntaxError: invalid syntax
>>> 
\end{verbatim}
\afterverb
%
There is little to be gained by arguing with Python.  It is a tool,
it has no emotion and it is happy and ready to serve you whenever you
need it.  Its error messages sound harsh, but they are just Python's
call for help.  It has looked at what you typed, and it simply cannot
understand what you have entered.

Python is much more like a dog, loving you unconditionally, having a few
key words that it understands, looking you with a sweet look on its
face ({\tt >>>}) and waiting for you to say something it understands.
When Python says ``SyntaxError: invalid syntax'', it is simply wagging
its tail and saying, ``You seemed to say something but I just don't
understand what you meant, but please keep talking to me ({\tt >>>}).''

As your programs become increasingly sophisticated, you will encounter three 
general types of errors:

\begin{description}

\item[Syntax errors:] These are the first errors you will make and the easiest
to fix.  A syntax error means that you have violated the ``grammar'' rules of Python.
Python does its best to point right at the line and character where 
it noticed it was confused.  The only tricky bit of syntax errors is that sometimes
the mistake that needs fixing is actually earlier in the program than where Python
{\em noticed} it was confused.  So the line and character that Python indicates in 
a syntax error may just be a starting point for your investigation.

\item[Logic errors:] A logic error is when your program has good syntax but there is a mistake 
in the order of the statements or perhaps a mistake in how the statements relate to one another.
A good example of a logic error might be, ``take a drink from your water bottle, put it 
in your backpack, walk to the library, and then put the top back on the bottle.''

\item[Semantic errors:] A semantic error is when your description of the steps to take 
is syntactically perfect and in the right order, but there is simply a mistake in 
the program.  The program is perfectly correct but it does not do what
you {\em intended} for it to do. A simple example would
be if you were giving a person directions to a restaurant and said, ``... when you reach
the intersection with the gas station, turn left and go one mile and the restaurant
is a red building on your left.''.  Your friend is very late and calls you to tell you that
they are on a farm and walking around behind a barn, with no sign of a restaurant.  
Then you say ``did you turn left or right at the gas station?'' and 
they say, ``I followed your directions perfectly, I have 
them written down, it says turn left and go one mile at the gas station.''.  Then you say,
``I am very sorry, because while my instructions were syntactically correct, they 
sadly contained a small but undetected semantic error.''. 

\end{description}

Again in all three types of errors, Python is merely trying its hardest to 
do exactly what you have asked.

\section{La jornada de aprendizaje}

As you progress through the rest of the book, don't be afraid if the concepts 
don't seem to fit together well the first time.  When you were learning to speak, 
it was not a problem  for your first few years that you just made cute gurgling noises.
And it was OK if it took six months for you to move from simple vocabulary to 
simple sentences and took 5-6 more years to move from sentences to paragraphs, and a
few more years to be able to write an interesting complete short story on your own.

We want you to learn Python much more rapidly, so we teach it all at the same time
over the next few chapters.  
But it is like learning a new language that takes time to absorb and understand
before it feels natural.
That leads to some confusion as we visit and revisit
topics to try to get you to see the big picture while we are defining the tiny
fragments that make up the big picture.  While the book is written linearly and
if you are taking a course, it will progress in a linear fashion, don't hesitate
to be very non-linear in how you approach the material.  Look forwards and backwards
and read with a light touch.  By skimming more advanced material without 
fully understanding the details, you can get a better understanding of the ``why?'' 
of programming.  By reviewing previous material and even re-doing earlier 
exercises, you will realize that you actually learned a lot of material even 
if the material you are currently staring at seems a bit impenetrable.

Usually when you are learning your first programming language, there are a few
wonderful ``Ah-Hah!'' moments where you can look up from pounding away at some rock
with a hammer and chisel and step away and see that you are indeed building 
a beautiful sculpture.

If something seems particularly hard, there is usually no value in staying up all 
night and staring at it.   Take a break, take a nap, have a snack, explain what you 
are having a problem with to someone (or perhaps your dog), and then come back to it with
fresh eyes.  I assure you that once you learn the programming concepts in the book
you will look back and see that it was all really easy and elegant and it simply 
took you a bit of time to absorb it.

\section{Glosario}

\begin{description}

\item[bug:]  An error in a program.
\index{bug}

\item[central processing unit:] The heart of any computer.  It is what
runs the software that we write; also called ``CPU'' or ``the processor''.
\index{central processing unit}
\index{CPU}

\item[compile:]  To translate a program written in a high-level language
into a low-level language all at once, in preparation for later
execution.
\index{compile}

\item[high-level language:]  A programming language like Python that
is designed to be easy for humans to read and write.
\index{high-level language}

\item[interactive mode:] A way of using the Python interpreter by
typing commands and expressions at the prompt.
\index{interactive mode}

\item[interpret:]  To execute a program in a high-level language
by translating it one line at a time.
\index{interpret}

\item[low-level language:]  A programming language that is designed
to be easy for a computer to execute; also called ``machine code'' or
``assembly language.''
\index{low-level language}

\item[machine code:]  The lowest level language for software which 
is the language that is directly executed by the central processing unit 
(CPU).
\index{machine code}

\item[main memory:] Stores programs and data.  Main memory loses 
its information when the power is turned off.
\index{main memory}

\item[parse:]  To examine a program and analyze the syntactic structure.
\index{parse}

\item[portability:]  A property of a program that can run on more
than one kind of computer.
\index{portability}

\item[print statement:]  An instruction that causes the Python
interpreter to display a value on the screen.
\index{print statement}
\index{statement!print}

\item[problem solving:]  The process of formulating a problem, finding
a solution, and expressing the solution.
\index{problem solving}

\item[program:] A set of instructions that specifies a computation.
\index{program}

\item[prompt:] When a program displays a message and pauses for the 
user to type some input to the program.
\index{prompt}

\item[secondary memory:] Stores programs and data and retains its 
information even when the power is turned off.  Generally slower 
than main memory.  Examples of secondary memory include disk 
drives and flash memory in USB sticks.
\index{secondary memory}

\item[semantics:]  The meaning of a program.
\index{semantics}

\item[semantic error:]   An error in a program that makes it do something
other than what the programmer intended.
\index{semantic error}

\item[source code:]  A program in a high-level language.
\index{source code}

\end{description}

\section{Ejercicios}


\begin{ex}
What is the function of the secondary memory in a computer?

a) Execute all of the computation and logic of the program\\
b) Retrieve web pages over the Internet\\
c) Store information for the long term - even beyond a power cycle\\
d) Take input from the user 
\end{ex}

\begin{ex}
?`Qu\'e es un programa?
\end{ex}

\begin{ex}
?`Cu\'al es la diferencia entre compilador e int\'erprete?
\end{ex}

\begin{ex}
?`Cu\'al de los siguientes contiene ''c\'odigo de m\'aquina'' (machine code)?

a) El int\'erprete Python\\
b) El teclado\\
c) Archivo fuente en Python\\
d) Un documento de procesador de palabras (word processing)
\end{ex}

\begin{ex}
?`Q\'ue hay de malo en el siguiente c\'odigo?:

\beforeverb
\begin{verbatim}
>>> primt 'Hello world!'
  File "<stdin>", line 1
    primt 'Hello world!'
                       ^
SyntaxError: invalid syntax
>>> 
\end{verbatim}
\afterverb

\end{ex}

\begin{ex}
?`D\'onde almacena la computadora una variable como "X" despu\'es de que  
se termina la siguiente l\'inea Python?

\beforeverb
\begin{verbatim}
x = 123
\end{verbatim}
\afterverb
%
a) Unidad central de procesamiento\\
b) Memoria Principal\\
c) Memoria Secundaria\\
d) Dispositivo de entrada (Input Device)\\
e) Dispositivo de salida (Output Device)
\end{ex}

\begin{ex}
?`Qu\'e imprime el siguiente programa?:

\beforeverb
\begin{verbatim}
x = 43
x = x + 1
print x
\end{verbatim}
\afterverb
%
a) 43\\
b) 44\\
c) x + 1\\
d) Error porque x = x + 1 no es posible matem\'aticamente
\end{ex}

\begin{ex}
Explique cada una de las siguientes utilizando un ejemplo de una capacidad humana: 
(1) Unidad Central de Procesamiento, (2) Memoria Principal, (3) Memoria Secundaria, 
(4) Dispositivo de entrada (Input Device), y 
(5) Dispositivo de salida (Output Device).
Por ejemplo, "?`Cu\'al es el equivalente humano de la Unidad Central de Procesamiento"? 
\end{ex}

\begin{ex}
?`C\'omo se arregla un error sint\'actico, "Syntax Error"?
\end{ex}
